\documentclass[11pt, reqno]{amsart}

\include{prelude.tex}

\usepackage{biblatex}
\usepackage{microtype}
\addbibresource{bibliography.bib}

\title{Project Proposal}
\author{Patrick Norton \and Elijah Whitlam-Sandler}

\begin{document}

\maketitle

\begin{abstract}
  % TODO
  This project is an attempt to classify the structure of certain mathematical
  objects. % TODO: Explain what mathematical objects
  The structure of these objects relates to important questions in both
  physics and mathematics. % TODO: What questions?
  We know that the structure of Lie algebras is completely controlled by
  specific elements, known as highest weight vectors. Our goal is to find all
  these highest weight vectors and determine how they relate to each other.
  % TODO: Prior art
\end{abstract}

\section{Background and Rationale}

This project is in the domain of Lie theory, which focuses around an object
called a \emph{Lie algebra}. A Lie algebra is a vector space with a binary
operation $[,]: \mathfrak{g} \times \mathfrak{g} \rightarrow \mathfrak{g}$ that satisfies certain nice properties~\cite{Erd06}.

An important structure related to Lie algebras is that of the Lie module. A Lie
module is a vector space that is ``acted on'' by a Lie algebra---that is, each
element of the algebra induces some transformation on the vector space in a
particularly nice way. This project focuses on one specific module, which we
notate $\Sym(\mathfrak{g})$.
% TODO: Explain Sym(g)

In addition to building up with Lie algebras, we can also break them down.
Similarly to how numbers decompose into primes, we can decompose Lie algebras
into subalgebras. A subalgebra of particular importance is the Cartan
subalgebra.

Like how algebras break down into subalgebras, we can break down modules into
submodules. This is the overall goal of our project: to break down $\Sym(\mathfrak{g})$
into submodules.

Important to our classification of submodules is the concept of a weight vector.
A weight vector is an element $v$ of a module such that everything that acts on
$v$ scales it by some constant factor. It turns out that being able to describe
these weight vectors is sufficient to be able to completely describe the
submodules of any module.

% TODO:
% Lie algebra
% Cartan subalgebra
% Weight vectors
% Highest weight vectors
% Modules & highest weight modules
% Relevant theorems (with citations)
% Allegedly, we know what the weights are in Sym(g)---what are they (w/source)
% All that physics stuff

\section{Specific Aim and Hypothesis}

% TODO:
% Explain what "classify" means

\section{Design and Procedure}

\section{Predicted and Alternative Outcomes}

\section{Role of Student}

The role of the students in this project will be to do computations, create examples, and prove theorems in order to build up to a classification of the highest weight vectors in $\Sym(\fg)$. The students will work together to produce (publish?) a paper (I assume this is the right word?) highlighting either a discovered classification, or any progress made towards such a result.

\section{Role of Faculty}

Zajj Daugherty and Noah Charles will (meetings of set frequency with research group?) be resources for the group and provide insight on good ways to proceed. % I have no clue what should actually go in this section

\section{Benefit to Student}

% TODO: Written by Zajj

\printbibliography{}

\end{document}
