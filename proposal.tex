\documentclass[11pt, reqno]{amsart}

\documentclass[11pt, reqno]{amsart}
\usepackage[margin=1in]{geometry}
\geometry{letterpaper}
%\geometry{landscape} % Activate for for rotated page geometry
\usepackage[parfill]{parskip} % Activate to begin paragraphs with an empty line rather than an indent
\usepackage{amsfonts, amscd, amssymb, amsthm, amsmath}
\usepackage{mathtools} %xmapsto etc
\usepackage{pdfsync} %leaves makers for tex searching
\usepackage{enumerate}
\usepackage{booktabs}


%%% Color %%%---------------------------------------------------------
\usepackage{color}
\usepackage[dvipsnames]{xcolor}
\definecolor{dred}{HTML}{C30101}
\definecolor{dorange}{rgb}{.9,.3,0}
\definecolor{dgrey}{rgb}{.4,.4,.4}
\definecolor{plumb}{HTML}{8105C1}\definecolor{alert}{HTML}{8105C1}
\definecolor{pumpkin}{HTML}{E47604}
\definecolor{rose}{HTML}{C10091}
\definecolor{dgreen}{HTML}{25A75B}
\definecolor{dblue}{HTML}{0066FF}
\definecolor{cornflower}{HTML}{3256C3}
\definecolor{viridian}{HTML}{099A97}
\definecolor{alert}{HTML}{3256C3}



\usepackage{hyperref} %[pdftex,bookmarks]
% \hypersetup{
%     colorlinks=true,
%     linkcolor=viridian,
%     filecolor=viridian,
%     citecolor=viridian,
%     urlcolor=viridian,
%     pdftitle={Overleaf Example},
%     pdfpagemode=FullScreen,
%     }

%%% Theorems %%%---------------------------------------------------------
\theoremstyle{plain}
\newtheorem{thm}{Theorem}[section]
\newtheorem{lemma}[thm]{Lemma}
\newtheorem{prop}[thm]{Proposition}
\newtheorem{cor}[thm]{Corollary}
\theoremstyle{definition}
\newtheorem*{def*}{Definition}
\newtheorem{defn}[thm]{Definition}
\newtheorem{topic}[thm]{Topic}
\theoremstyle{remark}
\newtheorem{example}{Example}[thm]
\newtheorem{remark}[thm]{Remark}
\newtheorem{subrem}[example]{Remark}


%%% Environments %%%---------------------------------------------------------
\newenvironment{pf}{\color{black}\medskip \paragraph*{\emph{Proof}.}}{\hfill \qedsymbol \medskip }
\newenvironment{ans}{\medskip \color{black} \paragraph*{\emph{Answer}.}}{\hfill \break $~\!\!$ \dotfill \medskip }
\newenvironment{sketch}{\medskip \paragraph*{\emph{Proof sketch}.}}{ \medskip }
\newenvironment{summary}{\medskip \paragraph*{\emph{Summary}.}}{ \hfill \break \rule{1.5cm}{0.4pt} \medskip }
\newcommand\Ans[1]{$ $\hfill {\color{black}\emph{Answer:} {#1}}}
\newcommand{\Hint}[1]{\hfill{\small [\emph{Hint:} {#1}]}}


%%% Color %%%---------------------------------------------------------
\usepackage{color}
\usepackage[dvipsnames]{xcolor}
\definecolor{dred}{HTML}{C30101}
\definecolor{dorange}{rgb}{.9,.3,0}
\definecolor{dgrey}{rgb}{.4,.4,.4}
\definecolor{plumb}{HTML}{8105C1}\definecolor{alert}{HTML}{8105C1}
\definecolor{pumpkin}{HTML}{E47604}
\definecolor{rose}{HTML}{C10091}
\definecolor{dgreen}{HTML}{25A75B}
\definecolor{dblue}{HTML}{0066FF}
\definecolor{cornflower}{HTML}{3256C3}
\definecolor{viridian}{HTML}{099A97}
\definecolor{alert}{HTML}{3256C3}


\newcommand\plumb[1]{{\color{plumb}#1}}
\newcommand\cornfl[1]{{\color{cornflower}#1}}
\newcommand\dgreen[1]{{\color{dgreen}#1}}
\newcommand\viridian[1]{{\color{viridian}#1}}
\newcommand\dblue[1]{{\color{dblue}#1}}
\newcommand\dred[1]{{\color{dred}#1}}
\newcommand\gray[1]{{\color{black!40}#1}}
\newcommand\black[1]{{\color{black}#1}}
\newcommand\pumpk[1]{{\color{pumpkin}#1}}
\newcommand\rose[1]{{\color{rose}#1}}



\newcommand{\NOTE}[1]{{\color{cornflower}#1}}
\newcommand{\blue}[1]{{\color{blue}#1}}
\newcommand{\red}[1]{{\color{red}#1}}
\newcommand{\alert}[1]{{\color{alert}#1}}
\newcommand{\Alert}[1]{\emph{\color{alert}#1}}


\usepackage{hyperref} %[pdftex,bookmarks]
% \hypersetup{
%     colorlinks=true,
%     linkcolor=viridian,
%     filecolor=viridian,
%     citecolor=viridian,
%     urlcolor=viridian,
%     pdftitle={Overleaf Example},
%     pdfpagemode=FullScreen,
%     }





%%% Pictures %%%---------------------------------------------------------
%%% If you need to draw pictures, tikzpicture is one good option. Here are some basic things I always use:
\usepackage{tikz}
\usetikzlibrary{scopes}
\usetikzlibrary{positioning}
\usepgflibrary{shapes}
\tikzstyle{V}=[draw, fill =black, circle, inner sep=0pt, minimum size=2pt]
\tikzstyle{bV}=[draw, fill =black, circle, inner sep=0pt, minimum size=4pt]
\tikzstyle{over}=[draw=white,double=black,line width=3pt]
\tikzstyle{C}=[draw, thick, fill =white, circle, inner sep=0pt, minimum size=6pt]


\newcommand\TikZ[1]{\begin{matrix}\begin{tikzpicture}#1\end{tikzpicture}\end{matrix}}

\newcounter{r}
\newcommand\Part[1]{
	\setcounter{r}{1}
	\foreach \x in {#1}{
			{\ifnum\value{r}=1
						\draw (0,\value{r}-1)--(\x,\value{r}-1);
					\fi}
			\draw (0,\value{r}) to (\x,\value{r});
			\foreach \y in {0, ..., \x} {\draw (\y,\value{r})--(\y,\value{r}-1);}
			\addtocounter{r}{1}
		}}
\def\PartUNIT{.175}
%Self-contained tikz images for \Part above.
\newcommand{\PART}[1]{
	\begin{matrix}
		\begin{tikzpicture}[xscale=\PartUNIT, yscale=-\PartUNIT]
			\Part{#1}
		\end{tikzpicture}
	\end{matrix}
}






%%% Alphabets %%%---------------------------------------------------------
%%% Some shortcuts for my commonly used special alphabets and characters.
\def\cA{\mathcal{A}}\def\cB{\mathcal{B}}\def\cC{\mathcal{C}}\def\cD{\mathcal{D}}\def\cE{\mathcal{E}}\def\cF{\mathcal{F}}\def\cG{\mathcal{G}}\def\cH{\mathcal{H}}\def\cI{\mathcal{I}}\def\cJ{\mathcal{J}}\def\cK{\mathcal{K}}\def\cL{\mathcal{L}}\def\cM{\mathcal{M}}\def\cN{\mathcal{N}}\def\cO{\mathcal{O}}\def\cP{\mathcal{P}}\def\cQ{\mathcal{Q}}\def\cR{\mathcal{R}}\def\cS{\mathcal{S}}\def\cT{\mathcal{T}}\def\cU{\mathcal{U}}\def\cV{\mathcal{V}}\def\cW{\mathcal{W}}\def\cX{\mathcal{X}}\def\cY{\mathcal{Y}}\def\cZ{\mathcal{Z}}

\def\AA{\mathbb{A}} \def\BB{\mathbb{B}} \def\CC{\mathbb{C}} \def\DD{\mathbb{D}} \def\EE{\mathbb{E}} \def\FF{\mathbb{F}} \def\GG{\mathbb{G}} \def\HH{\mathbb{H}} \def\II{\mathbb{I}} \def\JJ{\mathbb{J}} \def\KK{\mathbb{K}} \def\LL{\mathbb{L}} \def\MM{\mathbb{M}} \def\NN{\mathbb{N}} \def\OO{\mathbb{O}} \def\PP{\mathbb{P}} \def\QQ{\mathbb{Q}} \def\RR{\mathbb{R}} \def\SS{\mathbb{S}} \def\TT{\mathbb{T}} \def\UU{\mathbb{U}} \def\VV{\mathbb{V}} \def\WW{\mathbb{W}} \def\XX{\mathbb{X}} \def\YY{\mathbb{Y}} \def\ZZ{\mathbb{Z}}

\def\fa{\mathfrak{a}} \def\fb{\mathfrak{b}} \def\fc{\mathfrak{c}} \def\fd{\mathfrak{d}} \def\fe{\mathfrak{e}} \def\ff{\mathfrak{f}} \def\fg{\mathfrak{g}} \def\fh{\mathfrak{h}} \def\fj{\mathfrak{j}} \def\fk{\mathfrak{k}} \def\fl{\mathfrak{l}} \def\fm{\mathfrak{m}} \def\fn{\mathfrak{n}} \def\fo{\mathfrak{o}} \def\fp{\mathfrak{p}} \def\fq{\mathfrak{q}} \def\fr{\mathfrak{r}} \def\fs{\mathfrak{s}} \def\ft{\mathfrak{t}} \def\fu{\mathfrak{u}} \def\fv{\mathfrak{v}} \def\fw{\mathfrak{w}} \def\fx{\mathfrak{x}} \def\fy{\mathfrak{y}} \def\fz{\mathfrak{z}}
\def\fgl{\mathfrak{gl}}  \def\fsl{\mathfrak{sl}}  \def\fso{\mathfrak{so}}  \def\fsp{\mathfrak{sp}}
\def\fN{\mathfrak{N}}
\DeclareMathOperator{\GL}{GL}
\DeclareMathOperator{\SL}{SL}
\DeclareMathOperator{\SP}{SP}
\DeclareMathOperator{\OG}{O}

\def\aa{\mathbf{a}} \def\bb{\mathbf{b}} \def\cc{\mathbf{c}} \def\dd{\mathbf{d}} \def\ee{\mathbf{e}} \def\ff{\mathbf{f}}
%\def\gg{\mathbf{g}}
\def\hh{\mathbf{h}} \def\ii{\mathbf{i}} \def\jj{\mathbf{j}} \def\kk{\mathbf{k}}
%\def\ll{\mathbf{l}}
\def\mm{\mathbf{m}} \def\nn{\mathbf{n}} \def\oo{\mathbf{o}} \def\pp{\mathbf{p}} \def\qq{\mathbf{q}} \def\rr{\mathbf{r}} \def\ss{\mathbf{s}} \def\tt{\mathbf{t}} \def\uu{\mathbf{u}} \def\vv{\mathbf{v}} \def\ww{\mathbf{w}} \def\xx{\mathbf{x}} \def\yy{\mathbf{y}} \def\zz{\mathbf{z}}
\def\zzero{\mathbf{0}}

\def\<{\langle} \def\>{\rangle}
\DeclareMathOperator{\ad}{ad}
\DeclareMathOperator{\Aut}{Aut}
\DeclareMathOperator{\ch}{ch}
\DeclareMathOperator{\col}{col}
\DeclareMathOperator{\diag}{diag}
\DeclareMathOperator{\dimn}{dim}
\DeclareMathOperator{\End}{End}
\DeclareMathOperator{\ev}{ev}
\def\f{\varphi}
\def\half{\hbox{$\frac12$}}
\DeclareMathOperator{\Hom}{Hom}
\DeclareMathOperator{\img}{img}
\DeclareMathOperator{\Inn}{Inn}
\DeclareMathOperator{\id}{id}
\DeclareMathOperator{\rad}{rad}
\DeclareMathOperator{\Rep}{Rep}
\DeclareMathOperator{\row}{row}
\DeclareMathOperator{\rk}{rank}
\def\normeq{\trianglelefteq}
\DeclareMathOperator{\nul}{nullity}
\DeclareMathOperator{\sgn}{sgn}
\DeclareMathOperator{\spn}{span}
\DeclareMathOperator{\supp}{supp}
\DeclareMathOperator{\Syl}{Syl}
\DeclareMathOperator{\Sym}{Sym}
\DeclareMathOperator{\tr}{tr}
\def\vep{\varepsilon}
\DeclareMathOperator{\lcm}{lcm}

\def\ol{\overline}
\newcommand{\Mod}[1]{\ (\mathrm{mod}\ #1)}
\newcommand{\Wedge}[1]{\hbox{$\bigwedge\nolimits^{\! #1}$}}


\def\Hfill{$ $\hfill}


% Arrows:
\newcommand\xdhrightarrow[2][]{%
	\mathrel{\ooalign{$\xrightarrow[#1\mkern4mu]{#2\mkern4mu}$\cr%
			\hidewidth$\rightarrow\mkern4mu$}}
}
%\newcommand\dhrightarrow{%
% \mathrel{\ooalign{$\rightarrow$\cr%
% $\mkern3.5mu\rightarrow$}}
%}
\def\dhrightarrow{\twoheadrightarrow}
\def\dhleftarrow{\twoheadleftarrow}

\usepackage{mathabx}
\def\acts{\lefttorightarrow} %group action

\usepackage{wasysym}

% Arrays:
\newcommand\Pmatrix[1]{\begin{pmatrix}#1\end{pmatrix}}
\newcommand\smatrix[1]{\text{\small$\begin{pmatrix}#1\end{pmatrix}$}}
\newcommand\fmatrix[1]{\text{\footnotesize$\begin{pmatrix}#1\end{pmatrix}$}}
\newcommand\tmatrix[1]{\text{\tiny$\begin{pmatrix}#1\end{pmatrix}$}}

\DeclarePairedDelimiter{\norm}{\lVert}{\rVert}
\DeclarePairedDelimiter{\abs}{\lvert}{\rvert}
\DeclarePairedDelimiter{\ang}{\langle}{\rangle}
\DeclarePairedDelimiter\ceil{\lceil}{\rceil}
\DeclarePairedDelimiter\floor{\lfloor}{\rfloor}

\newcommand{\middlemid}{%
	\ensuremath{\;\middle\vert\;}
}

\newcommand{\dblang}[1]{%
	\ensuremath{\left\langle\!\left\langle#1\right\rangle\!\right\rangle}
}

\newcommand{\comment}[1]{%
	\text{\phantom{(#1)}} \tag{#1}%
}

\newcommand{\commath}[1]{%
	\phantom{(#1)} \tag{#1}%
}

\def\FOUR{4}\def\ONE{1}\def\FIVE{5}\def\EIGHT{8}\def\THREE{3}\def\TWO{2}\def\SIX{6}\def\SEVEN{7}

\newcommand{\chareq}{%
	\mathrel{\mathpalette\chRAW\relax}%
}


\makeatletter
\newcommand{\chRAW}[2]{%
	\sbox\z@{$#1\LHD$}%
	\sbox\tw@{$#1\leqslant$}%
	\dimen@=\ht\tw@
	\advance\dimen@-\ht\z@
	\advance\dimen@ .3pt
	\ifx#1\displaystyle
		\advance\dimen@ .2pt
	\else
		\ifx#1\textstyle
			\advance\dimen@ .2pt
		\fi
	\fi
	\ooalign{\raisebox{\dimen@}{$\m@th#1\LHD$}\cr$\m@th#1\leqslant$\cr}%
}
\makeatother


\usepackage[backend=biber, style=alphabetic]{biblatex}
\usepackage{microtype}
\addbibresource{bibliography.bib}

\title{Project Proposal}
\author{Patrick Norton \and Elijah Whitlam-Sandler \and Pranay Pingali}

\begin{document}

\maketitle

\begin{abstract}
  % TODO
  This project is an attempt to classify the structure of certain mathematical
  objects. % TODO: Explain what mathematical objects
  The structure of these objects relates to important questions in both
  physics and mathematics. % TODO: What questions?
  We know that the structure of Lie algebras is completely controlled by
  specific elements, known as highest weight vectors. Our goal is to find all
  these highest weight vectors and determine how they relate to each other.
  % TODO: Prior art
\end{abstract}

\section{Background and Rationale}

This project is in mathematical physics and representation theory. Specifically, we will be studying certain polynomials and their role in continuous phase transitions. A \emph{continuous phase transition} is a change to a system where some metric describing the system changes continuously over the transition \cite[\S1]{Sa06}. This is opposed to the general use of a phase transition where the phase changes may happen discretely. The classic example of a continuous phase transition is studying the magnetization---the strength of a magnet per unit volume---of a material as the material is heated. Applied heat causes the magnetization of a material to decrease continuously until the material is no longer magnetized. This is in contrast to a dis-continuous phase transition such as the boiling of water, where the density of the water changes discontinuously as the water changes from a liquid state to a gas state.

In order to study continuous phase transitions, 

USE TO TRANSITION TO MATH:\\
The highest weight vectors of a Lie Algebra $\fg$ in $V$ are exactly the eigenvectors 
of the Laplace-Casimir operator acting on $V$, so by studying the highest weight  
vectors of $\fg$, one is one is also studying properties of the Laplace-Casimir 
operator, which grants information about the behavior of continuous phase transitions. 
Studying highest weight vectors in $\Sym(\fg)$ allows us to narrow our search without
compromising the physical meaning, as the highest weight vectors in $\Sym(\fg)$ can be mapped back to the original function space. 



TRANSITION PHRASE INTO MATH
%The driving mathematical object in our work is a
 \emph{Lie algebra} $\fg$ (pronounced ``Lee''). This is a vector space, together with a non-associative ``multiplication'' called a \emph{Lie bracket}
\[[,]: \mathfrak{g} \times \mathfrak{g} \rightarrow \mathfrak{g}.\]
To each Lie algebra, we associate a set of polynomials, denoted $\Sym(\fg)$. This set is a special example of a Lie algebra \emph{module}, meaning that it is a vector space equipped with an \emph{action} by $\fg$: the elements of $\fg$ define linear functions from $\Sym(\fg)$ to itself. Our goal is to find those specific polynomials that generate the fundamental building blocks of our favorite $\fg$-module.


??? In addition to building up with Lie algebras, we can also break them down.
Similarly to how numbers decompose into primes, we can decompose Lie algebras
into subalgebras. And just like factoring large numbers is computationally difficult, we need special tools for decomposing large $\fg$-modules into their simple pieces. Each Lie algebra has a 
large family of pairwise commuting operators in $\fg$, called the \emph{Cartan} subalgebra, whose eigenvalues reveal the structure of any (reasonably nice) $\fg$-module. Our primary goal is to 
\begin{center} compute the eigenvalues and eigenvectors of the Cartan in $\Sym(\fg)$. \end{center}




%
%A subalgebra of particular importance is the Cartan
%subalgebra. Large family of pairwise commuting operators in $\fg$, whose eigenvalues reveal the structure of any (reasonably nice) $\fg$-module. 
%

%
%
%An important structure related to Lie algebras is that of the Lie module. A Lie
%module is a vector space that is ``acted on'' by a Lie algebra---that is, each
%element of the algebra induces some transformation on the vector space in a
%particularly nice way. This project focuses on one specific module, which we
%notate $\Sym(\mathfrak{g})$. 
%

\emph{Speaking technically}, assuming that $\fg$ is finite dimensional, $\Sym(\mathfrak{g})\coloneqq F\left[ \cB \right]$ 
is a polynomial ring where $F$ is the underlying field of $\fg$ and $\cB$ is a basis of $\fg$.



Like how algebras break down into subalgebras, we can break down modules into
submodules. This is the overall goal of our project: to break down $\Sym(\mathfrak{g})$
into submodules.

Important to our classification of submodules is the concept of a weight vector.
A weight vector is an element $v$ of a module such that everything that acts on
$v$ scales it by some constant factor. It turns out that being able to describe
these weight vectors is sufficient to be able to completely describe the
submodules of any module.

%These next two paragraphs about the physical implications need to be reviewed/rewritten, I had a hard time navigating the different spaces that were being discussed in Noah's project description, so there may be some inaccuracies.

%The study of the highest weight vectors of $\Sym(\fg)$ has implications in physics,
%mainly regarding the study of continuous phase transistions.


%The highest weight vectors of a Lie Algebra $\fg$ in $V$ are exactly the eigenvectors 
%of the Laplace-Casimir operator acting on $V$, so by studying the highest weight  
%vectors of $\fg$, one is one is also studying properties of the Laplace-Casimir 
%operator, which grants information about the behavior of continuous phase transitions. 
%Studying highest weight vectors in $\Sym(\fg)$ allows us to narrow our search without
%compromising the physical meaning, as the highest weight vectors in $\Sym(\fg)$ can be mapped back to the original function space. ***This section needs work, needs to go into more technical things (maybe? I'm not sure how much detail is necessary)***

% TODO:
% Lie algebra
% Cartan subalgebra
% Weight vectors
% Highest weight vectors
% Modules & highest weight modules
% Relevant theorems (with citations)
% Allegedly, we know what the weights are in Sym(g)---what are they (w/source)
% All that physics stuff

\section{Specific Aim and Hypothesis}

PUT MORE TECHNICAL DEFINITIONS HERE


This project aims to build on the theory regarding the highest weight vectors in $\Sym(\fg)$ by providing a classification of the highest weight vectors.  As a starting point, some work has been done on what the weights of $\Sym(\fg)$ are (***this needs citation, it was just mentioned in Zajj's email***). There has been additional work done on decomposing $\Sym(\fg)$ into invariant $\fg$-modules $\Sym(\fg)^\fg$ and a vector space $\cH$ of harmonic polynomials \cite[(3.10)]{NeRa2004}
\[\Sym(\fg)\cong\Sym(\fg)^\fg\otimes\cH.\] 

Our secondary goal is to be able to build bridges between representation theory, mathematical physics, and computational methods. There are many interesting results in representation theory that are not necessarily known to many physicists, and this can result in duplicated effort between the fields.

% \begin{itemize}
% \item Primary: Compute hw vectors
% \item Secondary: Build bridges between deeper representation theory literature/results, mathematical physics, and computational methods


% \end{itemize}




***This section needs work, I'm not sure what hypotheses can be made considering that we're not doing an experient, and I'm not sure how much further we can go into the specific aim, maybe discuss what is already known and some possible first steps, although that could also go in the next section***
% TODO:
% Explain what "classify" means

\section{Design and Procedure}




We'll begin by unpacking high level theorems about the decomposition of $\Sym(\fg)$ stated for general Lie algebras in terms of the three key examples pertinent to our study: the ``special linear'', ``special orthogonal'', and ``symplectic'' Lie algebras, notated $\fsl_n$, $\fso_n$, and $\fsp_n$ respectively.

CITE LECTURE NOTES and point out that this gives a decomp at a high level, but not in the practical concrete way we need.

STRENGTH: Using established knowledge of higher level mathematical structure to guide concrete computations, looking for specific examples. 

We plan to start by computing examples for a few of the smaller cases, in particular $\fg = \fsl_{n}$ ($n = 3, 4$), $\fsp_{6}$, and $\fso_{6}$. This will give us an opportunity to allow us to see the decomposition happen and give ideas as to possible classifications and help us determine how to proceed.

In addition to that, we plan to review the infinite families of weight vectors appearing in . % TODO: Cite

Once we have some idea of the structure, we plan to then generate code to derive all highest weight vectors of small degree. Having generated lots of examples, we should be able to use projection operators to build highest weight vectors in a more generalizable way. Once we have these, we should be able to leverage these results to build highest weight vectors in general degree. We plan to limit ourselves to vectors specific to the Lie algebras of types $A_{n}$, $C_{n}$, and $D_{n}$; this is because those are the types most applicable to continuous phase transitions.

% \begin{itemize}
% \item Do the first few small examples, seeing the decomposition happen for $\fg = \fsl_n$ ($n=3,4$), $\fsp_{6}$, $\fso_6$ in small degree homogeneous components.
% \item Review infinite families of weight vectors appearing in NOAH's THESIS.
% \item Generate code to derive all highest weight vectors in small degree.
% \item Use projection operators to build highest weight vectors in a more sophisticated way. (We know the weights)
% \item Leverage these results to build hw vecs in general degree (specific to types ACD)
% \end{itemize}





***Not sure what to put in this section since there is not really an experimental design or procedure part of our research.***

\section{Predicted and Alternative Outcomes}

Our predicted outcome is to find either recursive or closed formulas for the highest weight vectors of $\Sym(\fg)$, which will take the form of polynomials as $\Sym(\fg)$ is a polynomial ring. In the chance we are not able to achieve this goal, there are many meaningful contributions that can be made to the literature.

Much of the Lie theory literature that exists does so at a very high level, and offers little to no insight about the applictions of such theory. There is great value in being able to ground the currently available high level Lie theory so that it is more easily accessible to those without a background in representation theory who want to exploit the results currently out of reach.

It would be equally useful to add context to some of the current literature that is only truly available to experts in the field. There are many `results' that could be more accurately refered to as unwritten corollaries, which adding context to would allow the material to be used at a lower level.

There is also the outcome where we create computational methods based on previous theory. This would help increase the accessibility of the work on Lie algebras by making it easier for researchers in various fields to create working examples without necessarily needing to acquire a deep knowledge of the theory behind it.



% \begin{itemize}
% \item Predicted stuff: find either recursive or closed formulas for favorite polynomials.
% \item Unpack and ground mathematical theory for practical use. Build bridges from representation theory to physics.
% \item Add context to the literature that makes deep math theorems accessible. 
% \item Expand upon earlier progress with coding examples to produce computational tools.

% \end{itemize}

\section{Role of Student}

The role of the students in this project will be to do computations, create examples, and prove theorems in order to build up to a classification of the highest weight vectors in $\Sym(\fg)$. The students will work together to publish a paper highlighting either a discovered classification, or any progress made towards such a result. 

This project requires background knowledge of abstract algebra, and more specifically Lie theory. Patrick, Elijah and Pranay all have experience with groups, rings, and modules from taking Math 332 (\emph{Abstract Algebra}), and Patrick and Elijah have further experience with Lie algebras from the Fall 2023 section of Math 412 (\emph{Topics in Algebra: Lie Algebras}). Patrick and Elijah are also pursuing a Math-CS interdisciplinary major and a CS minor, respectively, which will aid them in writing code that can efficiently determine the highest weight vectors of a Lie algebra $\fg$, and Pranay is pursuing a physics major that will allow them to help draw meaningful connections between high level math and theoritical physics.


\section{Role of Faculty}

Zajj Daugherty and Noah Charles will (meetings of set frequency with research group?) be resources for the group and provide insight on good ways to proceed. ***Not sure about other roles***

\section{Benefit to Student}

% TODO: Written by Zajj

\printbibliography

\end{document}
