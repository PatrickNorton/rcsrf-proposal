\documentclass[11pt, reqno]{amsart}

\include{prelude.tex}

\usepackage{biblatex}
\usepackage{microtype}
\addbibresource{bibliography.bib}

\title{Project Proposal}
\author{Patrick Norton \and Elijah Whitlam-Sandler}

\begin{document}

\maketitle

\begin{abstract}
  % TODO
  This project is an attempt to classify the structure of certain mathematical
  objects. % TODO: Explain what mathematical objects
  The structure of these objects relates to important questions in both
  physics and mathematics. % TODO: What questions?
  We know that the structure of Lie algebras is completely controlled by
  specific elements, known as highest weight vectors. Our goal is to find all
  these highest weight vectors and determine how they relate to each other.
  % TODO: Prior art
\end{abstract}

\section{Background and Rationale}

This project is in the domain of Lie theory, which focuses around an object
called a \emph{Lie algebra}. A Lie algebra is a vector space with a binary
operation $[,]: \mathfrak{g} \times \mathfrak{g} \rightarrow \mathfrak{g}$ that satisfies certain nice properties~\cite{Erd06}.

An important structure related to Lie algebras is that of the Lie module. A Lie
module is a vector space that is ``acted on'' by a Lie algebra---that is, each
element of the algebra induces some transformation on the vector space in a
particularly nice way. This project focuses on one specific module, which we
notate $\Sym(\mathfrak{g})$. \emph{Speaking technically}, assuming that $\fg$ is finite dimensional, $\Sym(\mathfrak{g})\coloneqq F\left[ \cB \right]$ 
is a polynomial ring where $F$ is the underlying field of $\fg$ and $\cB$ is a basis of $\fg$.

In addition to building up with Lie algebras, we can also break them down.
Similarly to how numbers decompose into primes, we can decompose Lie algebras
into subalgebras. A subalgebra of particular importance is the Cartan
subalgebra.

Like how algebras break down into subalgebras, we can break down modules into
submodules. This is the overall goal of our project: to break down $\Sym(\mathfrak{g})$
into submodules.

Important to our classification of submodules is the concept of a weight vector.
A weight vector is an element $v$ of a module such that everything that acts on
$v$ scales it by some constant factor. It turns out that being able to describe
these weight vectors is sufficient to be able to completely describe the
submodules of any module.

%These next two paragraphs about the physical implications need to be reviewed/rewritten, I had a hard time navigating the different spaces that were being discussed in Noah's project description, so there may be some inaccuracies.

The study of the highest weight vectors of $\Sym(\fg)$ has implications in physics,
mainly regarding the study of continuous phase transistions. A continuous phase 
transition is a change to a system where some metric describing the system changes 
continuously over the transition.\cite{Sa06} This is opposed to the general use of a phase 
transition where the phase changes may happen discretely. The classic example of a 
continuous phase transition is studing the magnetization---the strength of a magnet 
per unit volume---of a material as the material is heated. Applied heat casuses the 
magnetization of a material to decrease continuously until the material is no longer 
magnetized. This is in contrast to a dis-continuous phase transition such as the 
boiling of water, where the density of the water changes discontinuously as the water 
changes from a liquid state to a gas state.

The highest weight vectors of a Lie Algebra $\fg$ in $V$ are exactly the eigenvectors 
of the Laplace-Casimir operator acting on $V$, so by studying the highest weight  
vectors of $\fg$, one is one is also studying properties of the Laplace-Casimir 
operator, which grants information bout the behavior of continuous phase transitions. 
Studying highest weight vectors in $\Sym(\fg)$ allows us to narrow our search without
comprimising the physical meaning, as the highest weight vectors in $\Sym(\fg)$ can be mapped back to the original function space. ***This section needs work, needs to go into more technical things (maybe? I'm not sure how much detail is necessary)***

% TODO:
% Lie algebra
% Cartan subalgebra
% Weight vectors
% Highest weight vectors
% Modules & highest weight modules
% Relevant theorems (with citations)
% Allegedly, we know what the weights are in Sym(g)---what are they (w/source)
% All that physics stuff

\section{Specific Aim and Hypothesis}

This project aims to build on the theory regarding the highest weight vectors in 
$\Sym(\fg)$ by providing a classification of the highest weight vectors. As a starting point, some work has been done on what the weights of $\Sym(\fg)$ are (***this needs citation, it was just mentioned in Zajj's email***). There has been additional work done on decomposing $\Sym(\fg)$ into invariant $\fg$-modules $\Sym(\fg)^\fg$ and a vector space $\cH$ of harmonic polynomials \cite[(3.10)]{NeRa2004}
\[\Sym(\fg)\cong\Sym(\fg)^\fg\otimes\cH.\]




***This section needs work, I'm not sure what hypotheses can be made considering that we're not doing an experient, and I'm not sure how much further we can go into the specific aim, maybe discuss what is already known and some possible first steps, although that could also go in the next section***
% TODO:
% Explain what "classify" means

\section{Design and Procedure}

***Not sure what to put in this section since there is not really an experimental design or procedure part of our research.***

\section{Predicted and Alternative Outcomes}

***Same for this section, not sure how we can predict outcomes besides saying that we're going to try and create a classification.***

\section{Role of Student}

The role of the students in this project will be to do computations, create examples, and prove theorems in order to build up to a classification of the highest weight vectors in $\Sym(\fg)$. The students will work together to publish a paper highlighting either a discovered classification, or any progress made towards such a result.

\section{Role of Faculty}

Zajj Daugherty and Noah Charles will (meetings of set frequency with research group?) be resources for the group and provide insight on good ways to proceed. ***Not sure about other roles***

\section{Benefit to Student}

% TODO: Written by Zajj

\printbibliography{}

\end{document}
